% Created 2019-09-19 jue 19:30
% Intended LaTeX compiler: pdflatex
\documentclass[11pt]{article}
\usepackage[utf8]{inputenc}
\usepackage[T1]{fontenc}
\usepackage{graphicx}
\usepackage{grffile}
\usepackage{longtable}
\usepackage{wrapfig}
\usepackage{rotating}
\usepackage[normalem]{ulem}
\usepackage{amsmath}
\usepackage{textcomp}
\usepackage{amssymb}
\usepackage{capt-of}
\usepackage{hyperref}
\author{O'conner Avelino}
\date{\today}
\title{Apuntes de gráficas}
\hypersetup{
 pdfauthor={O'conner Avelino},
 pdftitle={Apuntes de gráficas},
 pdfkeywords={},
 pdfsubject={},
 pdfcreator={Emacs 25.2.2 (Org mode 9.2)}, 
 pdflang={English}}
\begin{document}

\maketitle
\tableofcontents


\section{Defiinición de gráfica}
\label{sec:orge948146}

Una gráfica \(G\) consta de dos conjuntos, uno llamado el conjunto de
vértices y otro llamado el conjunto de aristas

\subsection{Vértices y aristas}
\label{sec:orgafe5795}

\subsection{Orden y tamaño}
\label{sec:orgfe3ad20}

\subsection{gráficas Regulares}
\label{sec:orgab2ec5c}

\subsubsection{Gráficas cubicas}
\label{sec:orgb092473}

\subsection{Isomorfismo de gráficas}
\label{sec:org483dccc}

\section{Complemento de una gráfica}
\label{sec:orgb0f45cb}

\section{Teorema fundamental}
\label{sec:org01672b5}
\end{document}